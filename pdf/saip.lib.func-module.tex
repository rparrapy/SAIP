%
% API Documentation for API Documentation
% Module saip.lib.func
%
% Generated by epydoc 3.0.1
% [Fri Jun 24 22:38:56 2011]
%

%%%%%%%%%%%%%%%%%%%%%%%%%%%%%%%%%%%%%%%%%%%%%%%%%%%%%%%%%%%%%%%%%%%%%%%%%%%
%%                          Module Description                           %%
%%%%%%%%%%%%%%%%%%%%%%%%%%%%%%%%%%%%%%%%%%%%%%%%%%%%%%%%%%%%%%%%%%%%%%%%%%%

    \index{saip \textit{(package)}!saip.lib \textit{(package)}!saip.lib.func \textit{(module)}|(}
\section{Module saip.lib.func}

    \label{saip:lib:func}
Módulo que provee funciones varias para la utilización en los 
controladores.

\textbf{Author:} \begin{itemize}
\setlength{\parskip}{0.6ex}
  \item Alejandro 
    Arce\footnote{\href{mailto:alearce07@gmail.com}{mailto:alearce07@gmail.com}}

  \item Gabriel 
    Caroni\footnote{\href{mailto:gabrielcaroni@gmail.com}{mailto:gabrielcaroni@gmail.com}}

  \item Rodrigo 
    Parra\footnote{\href{mailto:rodpar07@gmail.com}{mailto:rodpar07@gmail.com}}

\end{itemize}




%%%%%%%%%%%%%%%%%%%%%%%%%%%%%%%%%%%%%%%%%%%%%%%%%%%%%%%%%%%%%%%%%%%%%%%%%%%
%%                               Functions                               %%
%%%%%%%%%%%%%%%%%%%%%%%%%%%%%%%%%%%%%%%%%%%%%%%%%%%%%%%%%%%%%%%%%%%%%%%%%%%

  \subsection{Functions}

    \label{saip:lib:func:es_huerfano}
    \index{saip \textit{(package)}!saip.lib \textit{(package)}!saip.lib.func \textit{(module)}!saip.lib.func.es\_huerfano \textit{(function)}}

    \vspace{0.5ex}

\hspace{.8\funcindent}\begin{boxedminipage}{\funcwidth}

    \raggedright \textbf{es\_huerfano}(\textit{item})

    \vspace{-1.5ex}

    \rule{\textwidth}{0.5\fboxrule}
\setlength{\parskip}{2ex}
    Determina si un item es considerado huérfano o no.

\setlength{\parskip}{1ex}
      \textbf{Parameters}
      \vspace{-1ex}

      \begin{quote}
        \begin{Ventry}{xxxx}

          \item[item]

          Item que se analizará

            {\it (type=\texttt{Item})}

        \end{Ventry}

      \end{quote}

      \textbf{Return Value}
    \vspace{-1ex}

      \begin{quote}
      True si el item es huérfano, False en caso contrario.

      {\it (type=Bool)}

      \end{quote}

    \end{boxedminipage}

    \label{saip:lib:func:opuesto}
    \index{saip \textit{(package)}!saip.lib \textit{(package)}!saip.lib.func \textit{(module)}!saip.lib.func.opuesto \textit{(function)}}

    \vspace{0.5ex}

\hspace{.8\funcindent}\begin{boxedminipage}{\funcwidth}

    \raggedright \textbf{opuesto}(\textit{arista}, \textit{nodo})

    \vspace{-1.5ex}

    \rule{\textwidth}{0.5\fboxrule}
\setlength{\parskip}{2ex}
    Devuelve el otro item correspondiente a una relación

\setlength{\parskip}{1ex}
      \textbf{Parameters}
      \vspace{-1ex}

      \begin{quote}
        \begin{Ventry}{xxxxxx}

          \item[arista]

          Relación que se analizará

            {\it (type=\texttt{Relacion})}

          \item[nodo]

          Item conocido de la relación

            {\it (type=\texttt{Item})}

        \end{Ventry}

      \end{quote}

      \textbf{Return Value}
    \vspace{-1ex}

      \begin{quote}
      El otro item correspondiente a la relación, es decir, el que no se 
      recibió como parámetro

      {\it (type=\texttt{Item})}

      \end{quote}

    \end{boxedminipage}

    \label{saip:lib:func:relaciones_a_actualizadas}
    \index{saip \textit{(package)}!saip.lib \textit{(package)}!saip.lib.func \textit{(module)}!saip.lib.func.relaciones\_a\_actualizadas \textit{(function)}}

    \vspace{0.5ex}

\hspace{.8\funcindent}\begin{boxedminipage}{\funcwidth}

    \raggedright \textbf{relaciones\_a\_actualizadas}(\textit{aristas})

    \vspace{-1.5ex}

    \rule{\textwidth}{0.5\fboxrule}
\setlength{\parskip}{2ex}
    Provee la lista de relaciones actualizadas (con el hijo/sucesor en su 
    versión actual) a partir de una lista de relaciones .

\setlength{\parskip}{1ex}
      \textbf{Parameters}
      \vspace{-1ex}

      \begin{quote}
        \begin{Ventry}{xxxxxxx}

          \item[aristas]

          lista de relaciones

            {\it (type=list(\texttt{Relacion}))}

        \end{Ventry}

      \end{quote}

      \textbf{Return Value}
    \vspace{-1ex}

      \begin{quote}
      lista filtrada de relaciones con el item hijo/sucesor en su en su 
      versión actual

      {\it (type=list(\texttt{Relacion}))}

      \end{quote}

    \end{boxedminipage}

    \label{saip:lib:func:relaciones_b_actualizadas}
    \index{saip \textit{(package)}!saip.lib \textit{(package)}!saip.lib.func \textit{(module)}!saip.lib.func.relaciones\_b\_actualizadas \textit{(function)}}

    \vspace{0.5ex}

\hspace{.8\funcindent}\begin{boxedminipage}{\funcwidth}

    \raggedright \textbf{relaciones\_b\_actualizadas}(\textit{aristas})

    \vspace{-1.5ex}

    \rule{\textwidth}{0.5\fboxrule}
\setlength{\parskip}{2ex}
    Provee la lista de relaciones actualizadas (con el padre/antecesor en 
    su versión actual) a partir de una lista de relaciones .

\setlength{\parskip}{1ex}
      \textbf{Parameters}
      \vspace{-1ex}

      \begin{quote}
        \begin{Ventry}{xxxxxxx}

          \item[aristas]

          lista de relaciones

            {\it (type=list(\texttt{Relacion}))}

        \end{Ventry}

      \end{quote}

      \textbf{Return Value}
    \vspace{-1ex}

      \begin{quote}
      lista filtrada de relaciones con el item hijo/sucesor en su en su 
      versión actual

      {\it (type=list(\texttt{Relacion}))}

      \end{quote}

    \end{boxedminipage}

    \label{saip:lib:func:relaciones_a_recuperar}
    \index{saip \textit{(package)}!saip.lib \textit{(package)}!saip.lib.func \textit{(module)}!saip.lib.func.relaciones\_a\_recuperar \textit{(function)}}

    \vspace{0.5ex}

\hspace{.8\funcindent}\begin{boxedminipage}{\funcwidth}

    \raggedright \textbf{relaciones\_a\_recuperar}(\textit{aristas})

    \vspace{-1.5ex}

    \rule{\textwidth}{0.5\fboxrule}
\setlength{\parskip}{2ex}
    Provee la lista de relaciones sin duplicados (con el hijo/sucesor en la
    mayor versión presente en la lista recibida, por cada id\_item) a 
    partir de una lista de relaciones .

\setlength{\parskip}{1ex}
      \textbf{Parameters}
      \vspace{-1ex}

      \begin{quote}
        \begin{Ventry}{xxxxxxx}

          \item[aristas]

          lista de relaciones

            {\it (type=list(\texttt{Relacion}))}

        \end{Ventry}

      \end{quote}

      \textbf{Return Value}
    \vspace{-1ex}

      \begin{quote}
      lista filtrada de relaciones con el item hijo/sucesor en la mayor 
      versión presente en aristas

      {\it (type=list(\texttt{Relacion}))}

      \end{quote}

    \end{boxedminipage}

    \label{saip:lib:func:relaciones_b_recuperar}
    \index{saip \textit{(package)}!saip.lib \textit{(package)}!saip.lib.func \textit{(module)}!saip.lib.func.relaciones\_b\_recuperar \textit{(function)}}

    \vspace{0.5ex}

\hspace{.8\funcindent}\begin{boxedminipage}{\funcwidth}

    \raggedright \textbf{relaciones\_b\_recuperar}(\textit{aristas})

    \vspace{-1.5ex}

    \rule{\textwidth}{0.5\fboxrule}
\setlength{\parskip}{2ex}
    Provee la lista de relaciones sin duplicados (con el padre/antecesor en
    la mayor versión presente en la lista recibida, por cada id\_item) a 
    partir de una lista de relaciones .

\setlength{\parskip}{1ex}
      \textbf{Parameters}
      \vspace{-1ex}

      \begin{quote}
        \begin{Ventry}{xxxxxxx}

          \item[aristas]

          lista de relaciones

            {\it (type=list(\texttt{Relacion}))}

        \end{Ventry}

      \end{quote}

      \textbf{Return Value}
    \vspace{-1ex}

      \begin{quote}
      lista filtrada de relaciones con el item padre/antecesor en la mayor 
      versión presente en aristas

      {\it (type=list(\texttt{Relacion}))}

      \end{quote}

    \end{boxedminipage}

    \label{saip:lib:func:forma_ciclo}
    \index{saip \textit{(package)}!saip.lib \textit{(package)}!saip.lib.func \textit{(module)}!saip.lib.func.forma\_ciclo \textit{(function)}}

    \vspace{0.5ex}

\hspace{.8\funcindent}\begin{boxedminipage}{\funcwidth}

    \raggedright \textbf{forma\_ciclo}(\textit{nodo}, \textit{nodos\_explorados}={\tt []}, \textit{aristas\_exploradas}={\tt []}, \textit{band}={\tt False}, \textit{nivel}={\tt 1})

    \vspace{-1.5ex}

    \rule{\textwidth}{0.5\fboxrule}
\setlength{\parskip}{2ex}
    Determina recursivamente si existe un ciclo en la componente conexa del
    grafo de items a la que pertenece un item dado.

\setlength{\parskip}{1ex}
      \textbf{Parameters}
      \vspace{-1ex}

      \begin{quote}
        \begin{Ventry}{xxxxxxxxxxxxxxxxxx}

          \item[nodo]

          Item dado.(Normalmente acaba de añadírsele un relación)

            {\it (type=\{Item\})}

          \item[nodos\_explorados]

          Lista de items que ya han sido visitados.

            {\it (type=list(\{Item\}))}

          \item[aristas\_exploradas]

          Lista de relaciones visitadas.

            {\it (type=list(\{Relacion\}))}

          \item[band]

          Bandera que indica si ya se ha encontrado un bucle.

            {\it (type=Bool)}

          \item[nivel]

          Indica la cantidad de llamadas recursivas anidadas realizadas.

            {\it (type=Integer)}

        \end{Ventry}

      \end{quote}

      \textbf{Return Value}
    \vspace{-1ex}

      \begin{quote}
      True si existe un bucle, False en caso contrario.

      {\it (type=Bool)}

      \end{quote}

    \end{boxedminipage}

    \label{saip:lib:func:color}
    \index{saip \textit{(package)}!saip.lib \textit{(package)}!saip.lib.func \textit{(module)}!saip.lib.func.color \textit{(function)}}

    \vspace{0.5ex}

\hspace{.8\funcindent}\begin{boxedminipage}{\funcwidth}

    \raggedright \textbf{color}(\textit{nodo})

    \vspace{-1.5ex}

    \rule{\textwidth}{0.5\fboxrule}
\setlength{\parskip}{2ex}
    Determina el color que debe tener un item para la representación 
    gráfica del costo de impacto basado en el orden de la fase a la que 
    pertenece.

\setlength{\parskip}{1ex}
      \textbf{Parameters}
      \vspace{-1ex}

      \begin{quote}
        \begin{Ventry}{xxxx}

          \item[nodo]

          Item a colorear.

            {\it (type=\{Item\})}

        \end{Ventry}

      \end{quote}

      \textbf{Return Value}
    \vspace{-1ex}

      \begin{quote}
      El color que debe usarse para colorear el item.

      {\it (type=String)}

      \end{quote}

    \end{boxedminipage}

    \label{saip:lib:func:costo_impacto}
    \index{saip \textit{(package)}!saip.lib \textit{(package)}!saip.lib.func \textit{(module)}!saip.lib.func.costo\_impacto \textit{(function)}}

    \vspace{0.5ex}

\hspace{.8\funcindent}\begin{boxedminipage}{\funcwidth}

    \raggedright \textbf{costo\_impacto}(\textit{nodo}, \textit{grafo}, \textit{nodos\_explorados}={\tt []}, \textit{aristas\_exploradas}={\tt []}, \textit{costo}={\tt 0})

    \vspace{-1.5ex}

    \rule{\textwidth}{0.5\fboxrule}
\setlength{\parskip}{2ex}
    Calcula recursivamente el costo de impacto de un item determinado y 
    genera el grafo para la representación gráfica del resultado.

\setlength{\parskip}{1ex}
      \textbf{Parameters}
      \vspace{-1ex}

      \begin{quote}
        \begin{Ventry}{xxxxxxxxxxxxxxxxxx}

          \item[nodo]

          Item dado

            {\it (type=\{Item\})}

          \item[grafo]

          Grafo que se utilirá para la representación gráfica de 
          resultados.

            {\it (type=grafo Pydot)}

          \item[nodos\_explorados]

          Lista de items que ya han sido visitados.

            {\it (type=list(\{Item\}))}

          \item[aristas\_exploradas]

          Lista de relaciones visitadas.

            {\it (type=list(\{Relacion\}))}

          \item[costo]

          Suma de las complejidades de los items visitados.

            {\it (type=Integer)}

        \end{Ventry}

      \end{quote}

      \textbf{Return Value}
    \vspace{-1ex}

      \begin{quote}
      El costo de impacto y el grafo para representar el resultado.

      \end{quote}

    \end{boxedminipage}

    \label{saip:lib:func:estado_proyecto}
    \index{saip \textit{(package)}!saip.lib \textit{(package)}!saip.lib.func \textit{(module)}!saip.lib.func.estado\_proyecto \textit{(function)}}

    \vspace{0.5ex}

\hspace{.8\funcindent}\begin{boxedminipage}{\funcwidth}

    \raggedright \textbf{estado\_proyecto}(\textit{proyecto})

    \vspace{-1.5ex}

    \rule{\textwidth}{0.5\fboxrule}
\setlength{\parskip}{2ex}
    Asigna el estado al correspondiente a un proyecto dado.

\setlength{\parskip}{1ex}
      \textbf{Parameters}
      \vspace{-1ex}

      \begin{quote}
        \begin{Ventry}{xxxxxxxx}

          \item[proyecto]

          Proyecto cuyo estado desea actualizarse.

            {\it (type=\{Proyecto\})}

        \end{Ventry}

      \end{quote}

    \end{boxedminipage}

    \label{saip:lib:func:estado_fase}
    \index{saip \textit{(package)}!saip.lib \textit{(package)}!saip.lib.func \textit{(module)}!saip.lib.func.estado\_fase \textit{(function)}}

    \vspace{0.5ex}

\hspace{.8\funcindent}\begin{boxedminipage}{\funcwidth}

    \raggedright \textbf{estado\_fase}(\textit{fase})

    \vspace{-1.5ex}

    \rule{\textwidth}{0.5\fboxrule}
\setlength{\parskip}{2ex}
    Asigna el estado correspondiente a una fase dada.

\setlength{\parskip}{1ex}
      \textbf{Parameters}
      \vspace{-1ex}

      \begin{quote}
        \begin{Ventry}{xxxx}

          \item[fase]

          Fase cuyo estado desea actualizarse.

            {\it (type=\{Fase\})}

        \end{Ventry}

      \end{quote}

    \end{boxedminipage}

    \label{saip:lib:func:sucesor}
    \index{saip \textit{(package)}!saip.lib \textit{(package)}!saip.lib.func \textit{(module)}!saip.lib.func.sucesor \textit{(function)}}

    \vspace{0.5ex}

\hspace{.8\funcindent}\begin{boxedminipage}{\funcwidth}

    \raggedright \textbf{sucesor}(\textit{item})

    \vspace{-1.5ex}

    \rule{\textwidth}{0.5\fboxrule}
\setlength{\parskip}{2ex}
    Evalúa si un item tiene o no un sucesor en la fase siguiente.

\setlength{\parskip}{1ex}
      \textbf{Parameters}
      \vspace{-1ex}

      \begin{quote}
        \begin{Ventry}{xxxx}

          \item[item]

          Item dado

            {\it (type=\{Item\})}

        \end{Ventry}

      \end{quote}

      \textbf{Return Value}
    \vspace{-1ex}

      \begin{quote}
      True si cuenta con un sucesor, False en caso contrario.

      \end{quote}

    \end{boxedminipage}

    \label{saip:lib:func:consistencia_lb}
    \index{saip \textit{(package)}!saip.lib \textit{(package)}!saip.lib.func \textit{(module)}!saip.lib.func.consistencia\_lb \textit{(function)}}

    \vspace{0.5ex}

\hspace{.8\funcindent}\begin{boxedminipage}{\funcwidth}

    \raggedright \textbf{consistencia\_lb}(\textit{lb})

    \vspace{-1.5ex}

    \rule{\textwidth}{0.5\fboxrule}
\setlength{\parskip}{2ex}
    Evalúa la consistencia de una línea base y asigna el valor 
    correspondiente.

\setlength{\parskip}{1ex}
      \textbf{Parameters}
      \vspace{-1ex}

      \begin{quote}
        \begin{Ventry}{xx}

          \item[lb]

          Linea base a evaluar.

            {\it (type=\texttt{LineaBase})}

        \end{Ventry}

      \end{quote}

    \end{boxedminipage}

    \label{saip:lib:func:proximo_id}
    \index{saip \textit{(package)}!saip.lib \textit{(package)}!saip.lib.func \textit{(module)}!saip.lib.func.proximo\_id \textit{(function)}}

    \vspace{0.5ex}

\hspace{.8\funcindent}\begin{boxedminipage}{\funcwidth}

    \raggedright \textbf{proximo\_id}(\textit{lista\_ids})

    \vspace{-1.5ex}

    \rule{\textwidth}{0.5\fboxrule}
\setlength{\parskip}{2ex}
    Determina el siguiente id a ser utilizado para un objeto del sistema 
    (proyecto, fase, item, etc.).

\setlength{\parskip}{1ex}
      \textbf{Parameters}
      \vspace{-1ex}

      \begin{quote}
        \begin{Ventry}{xxxxxxxxx}

          \item[lista\_ids]

          lista de ids existentes del elemento del modelo dado.

            {\it (type=list())}

        \end{Ventry}

      \end{quote}

      \textbf{Return Value}
    \vspace{-1ex}

      \begin{quote}
      proximo id a ser utilizado.

      \end{quote}

    \end{boxedminipage}

    \index{saip \textit{(package)}!saip.lib \textit{(package)}!saip.lib.func \textit{(module)}|)}
