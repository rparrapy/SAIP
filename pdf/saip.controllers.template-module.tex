%
% API Documentation for API Documentation
% Module saip.controllers.template
%
% Generated by epydoc 3.0.1
% [Fri Jun 24 22:38:56 2011]
%

%%%%%%%%%%%%%%%%%%%%%%%%%%%%%%%%%%%%%%%%%%%%%%%%%%%%%%%%%%%%%%%%%%%%%%%%%%%
%%                          Module Description                           %%
%%%%%%%%%%%%%%%%%%%%%%%%%%%%%%%%%%%%%%%%%%%%%%%%%%%%%%%%%%%%%%%%%%%%%%%%%%%

    \index{saip \textit{(package)}!saip.controllers \textit{(package)}!saip.controllers.template \textit{(module)}|(}
\section{Module saip.controllers.template}

    \label{saip:controllers:template}
Fallback controller.


%%%%%%%%%%%%%%%%%%%%%%%%%%%%%%%%%%%%%%%%%%%%%%%%%%%%%%%%%%%%%%%%%%%%%%%%%%%
%%                           Class Description                           %%
%%%%%%%%%%%%%%%%%%%%%%%%%%%%%%%%%%%%%%%%%%%%%%%%%%%%%%%%%%%%%%%%%%%%%%%%%%%

    \index{saip \textit{(package)}!saip.controllers \textit{(package)}!saip.controllers.template \textit{(module)}!saip.controllers.template.TemplateController \textit{(class)}|(}
\subsection{Class TemplateController}

    \label{saip:controllers:template:TemplateController}
\begin{tabular}{cccccccc}
% Line for tg.TGController, linespec=[False, False]
\multicolumn{2}{r}{\settowidth{\BCL}{tg.TGController}\multirow{2}{\BCL}{tg.TGController}}
&&
&&
  \\\cline{3-3}
  &&\multicolumn{1}{c|}{}
&&
&&
  \\
% Line for saip.lib.base.BaseController, linespec=[False]
\multicolumn{4}{r}{\settowidth{\BCL}{saip.lib.base.BaseController}\multirow{2}{\BCL}{saip.lib.base.BaseController}}
&&
  \\\cline{5-5}
  &&&&\multicolumn{1}{c|}{}
&&
  \\
&&&&\multicolumn{2}{l}{\textbf{saip.controllers.template.TemplateController}}
\end{tabular}

The fallback controller for SAIP.

By default, the final controller tried to fulfill the request when no other
routes match. It may be used to display a template when all else fails, 
e.g.:

\begin{alltt}
   def view(self, url):
       return render('/\%s' \% url)\end{alltt}

Or if you're using Mako and want to explicitly send a 404 (Not Found) 
response code when the requested template doesn't exist:

\begin{alltt}
   import mako.exceptions
   
   def view(self, url):
       try:
           return render('/\%s' \% url)
       except mako.exceptions.TopLevelLookupException:
           abort(404)\end{alltt}


%%%%%%%%%%%%%%%%%%%%%%%%%%%%%%%%%%%%%%%%%%%%%%%%%%%%%%%%%%%%%%%%%%%%%%%%%%%
%%                                Methods                                %%
%%%%%%%%%%%%%%%%%%%%%%%%%%%%%%%%%%%%%%%%%%%%%%%%%%%%%%%%%%%%%%%%%%%%%%%%%%%

  \subsubsection{Methods}

    \label{saip:controllers:template:TemplateController:view}
    \index{saip \textit{(package)}!saip.controllers \textit{(package)}!saip.controllers.template \textit{(module)}!saip.controllers.template.TemplateController \textit{(class)}!saip.controllers.template.TemplateController.view \textit{(method)}}

    \vspace{0.5ex}

\hspace{.8\funcindent}\begin{boxedminipage}{\funcwidth}

    \raggedright \textbf{view}(\textit{self}, \textit{url})

    \vspace{-1.5ex}

    \rule{\textwidth}{0.5\fboxrule}
\setlength{\parskip}{2ex}
    Abort the request with a 404 HTTP status code.

\setlength{\parskip}{1ex}
    \end{boxedminipage}


\large{\textbf{\textit{Inherited from saip.lib.base.BaseController\textit{(Section \ref{saip:lib:base:BaseController})}}}}

\begin{quote}
\_\_call\_\_()
\end{quote}
    \index{saip \textit{(package)}!saip.controllers \textit{(package)}!saip.controllers.template \textit{(module)}!saip.controllers.template.TemplateController \textit{(class)}|)}
    \index{saip \textit{(package)}!saip.controllers \textit{(package)}!saip.controllers.template \textit{(module)}|)}
